\documentclass[a4paper,twoside,openany,CJK]{ctexbook}
\usepackage{geometry}

\newcommand{\citeu}[1]{$^{\mbox{\protect \scriptsize \cite{#1}}}$}
%\usepackage{amssymb}
%\usepackage{amsmath}
\usepackage{graphicx}
%\usepackage{bm}
\usepackage{epsf}
\usepackage{ifxetex}
\usepackage[numbers,sort&compress]{natbib}% 此命令要放在设置hyperref的命令之前
\usepackage{graphicx}  % needed for figures
\usepackage{float}
\usepackage{dcolumn}   % needed for some tables
\usepackage{bm}        % for math
\usepackage{amssymb}   % for math - conflict with stix
\usepackage{slashed}   % for Dirac Slash
\usepackage{amsmath}   % for mutiline eqn
\usepackage{simplewick} % for contraction
\usepackage{verbatim}  % for multi-line comment
\usepackage{color} % for textcolor
\usepackage{appendix} % for appendix
\usepackage{subfig}
\usepackage{makeidx}
\usepackage{mathrsfs}
\usepackage{listings}
\usepackage{xcolor}

% special symbols
\usepackage{manfnt}
%\usepackage{cypriot}
%\usepackage{stix}

%\usepackage{amsmath}
%\usepackage{amssymb}
%\usepackage{ctex}  %就是在这里引入了ctex宏包,下面就可以直接输入中文编译之后就可以正常输出啦

\usepackage{algorithm,algpseudocode}

\hyphenation{ALPGEN}
\hyphenation{EVTGEN}
\hyphenation{PYTHIA}

\usepackage[bookmarks=true,colorlinks=true,linkcolor=blue,unicode=true]{hyperref}

\usepackage{titlesec}

\titleclass{\subsubsubsection}{straight}[\subsection]

\newcounter{subsubsubsection}[subsubsection]
\renewcommand\thesubsubsubsection{\thesubsubsection.\arabic{subsubsubsection}}
\renewcommand\theparagraph{\thesubsubsubsection.\arabic{paragraph}} % optional; useful if paragraphs are to be numbered

\titleformat{\subsubsubsection}
  {\normalfont\normalsize\bfseries}{\thesubsubsubsection}{1em}{}
\titlespacing*{\subsubsubsection}
{0pt}{3.25ex plus 1ex minus .2ex}{1.5ex plus .2ex}

\makeatletter
\renewcommand\paragraph{\@startsection{paragraph}{5}{\z@}%
  {3.25ex \@plus1ex \@minus.2ex}%
  {-1em}%
  {\normalfont\normalsize\bfseries}}
\renewcommand\subparagraph{\@startsection{subparagraph}{6}{\parindent}%
  {3.25ex \@plus1ex \@minus .2ex}%
  {-1em}%
  {\normalfont\normalsize\bfseries}}
\def\toclevel@subsubsubsection{4}
\def\toclevel@paragraph{5}
\def\toclevel@paragraph{6}
\def\l@subsubsubsection{\@dottedtocline{4}{7em}{4em}}
\def\l@paragraph{\@dottedtocline{5}{10em}{5em}}
\def\l@subparagraph{\@dottedtocline{6}{14em}{6em}}
\makeatother

\setcounter{secnumdepth}{4}
\setcounter{tocdepth}{4}

\zihao{4}
\title{Learning Panda Quantum toolkit}
\author{Ji-Chong Yang}
\date{}

\makeindex

\begin{document}

\maketitle

\clearpage

\tableofcontents

\clearpage

\lstset{
    numbers=left,
    numberstyle= \tiny,
    keywordstyle= \color{ blue!70},
    commentstyle= \color{red!50!green!50!blue!50},
    frame=single,
    rulesepcolor= \color{ red!20!green!20!blue!20} ,
    escapeinside=``,
    xleftmargin=2em,xrightmargin=2em, aboveskip=1em,
    framexleftmargin=2em,
    language=c++,
    breaklines=true,
    columns=fullflexible,
    captionpos=b,
    basicstyle=\footnotesize\ttfamily,
}

由于对本领域并不熟悉,本文档会夹杂中文。

\chapter{\label{sec:BasicTopics}Basic Topics}

\section{\label{sec:BuildingBlocks}Building blocks}

\subsection{\label{sec:PhaseKickback}Phase kickback}

Phase kickback是很多算法中用到的一个技术。比如QFT, QPE,以及一些量子模拟算法。

首先举两个例子。一个是CNOT。

\begin{equation}
\begin{split}
&CNOT|00\rangle = |00\rangle, CNOT|01\rangle = |01\rangle, CNOT|10\rangle = |11\rangle, CNOT|11\rangle = |10\rangle
\end{split}
\end{equation}
因此
\begin{equation}
\begin{split}
&CNOT\left(|i0\rangle - |i1\rangle\right)=(-1)^i \left(|i0\rangle - |i1\rangle\right)
\end{split}
\end{equation}

也就是对第二个qubit的操作其实并没有做任何事情,只是把第一个qubit放到了相位上。

类似的$C-U$,其中$U$是作用到第二个qubit上的,并且假定第二个qubit是$U$的本征态,结果是
\begin{equation}
\begin{split}
&C-U\left(|0\rangle + |1\rangle\right)|u\rangle =\left(|0u\rangle + u|1u\rangle\right)=(|0\rangle+u|1\rangle)|u\rangle
\end{split}
\end{equation}

这件事可以用来做量子模拟,为了方便讨论$\sigma _z$

首先注意到
\begin{equation}
\begin{split}
&e^{ia_0\sigma _{z_1}} \cdot e^{ib_0\sigma _{z_2}}=e^{ia_0\sigma _{z_1} \otimes b_0\sigma _{z_2}}\\
&e^{ia_0\sigma _{z_1} \otimes b_0\sigma _{z_2}} = e^{ia_0 b_0 \sigma _{z_1} \otimes \sigma _{z_2}}\\
\end{split}
\end{equation}
所以我们可以专注于$e^{ic\sigma _{z_1}\otimes \sigma _{z_2}}$


\begin{equation}
\begin{split}
&e^{ic\sigma _{z1}\otimes \sigma _{z2}}\left(a_1|0\rangle + b_1|1\rangle\right)\left(a_2|0\rangle + b_2|1\rangle\right)|0\rangle \\
&=e^{ic\sigma _{z1}\otimes \sigma _{z2}}\left(a_1 a_2|00\rangle + b_1 a_2|10\rangle + a_1b_2|01\rangle + b_1b_2|11\rangle\right)|0\rangle \\
&=\left(e^{ic 1 \times 1}a_1 a_2 |00\rangle + e^{ic (-1)\times 1}b_1 a_2|10\rangle + e^{ic 1 \times (-1)}a_1b_2|01\rangle + e^{ic (-1)\times (-1)}b_1b_2|11\rangle\right)|0\rangle \\
\end{split}
\end{equation}

所以
\begin{equation}
\begin{split}
&CNOT_1 CNOT_2 e^{ic\sigma _{z3}}  CNOT_2 CNOT_1 \left(a_1 a_2|00\rangle + b_1 a_2|10\rangle + a_1b_2|01\rangle + b_1b_2|11\rangle\right)|0\rangle \\
&= CNOT_1 CNOT_2 e^{ic\sigma _{z3}}  CNOT_2 \left(a_1 a_2|000\rangle + b_1 a_2|101\rangle + a_1b_2|010\rangle + b_1b_2|111\rangle\right)\\
&= CNOT_1 CNOT_2 e^{ic\sigma _{z3}}  \left(a_1 a_2|000\rangle + b_1 a_2|101\rangle + a_1b_2|011\rangle + b_1b_2|110\rangle\right)\\
&= CNOT_1 CNOT_2  \left(e^{ic}a_1 a_2|000\rangle + e^{-ic}b_1 a_2|101\rangle + e^{-ic}a_1b_2|011\rangle + e^{ic}b_1b_2|110\rangle\right)\\
&= \left(e^{ic}a_1 a_2|00\rangle + e^{-ic}b_1 a_2|10\rangle + e^{-ic}a_1b_2|01\rangle + e^{ic}b_1b_2|11\rangle\right)|0\rangle\\
\end{split}
\end{equation}


怎么做一个$\sigma _{x,y}$的模拟呢?注意到$\sigma _x = H\cdot \sigma _z \cdot H$和,$\sigma _y = p(\frac{\pi}{2})\cdot H\cdot \sigma _z \cdot H \cdot p(-\frac{\pi}{2})$。

\chapter{\label{sec:Input}Input}

这一章专注于如何将经典数据翻译到量子线路中。

这一章分为两个部分,一部分是怎么把一个经典的矩阵变成量子线路。一部分是怎么把经典的矢量变成量子态。

\section{\label{sec:ZYZ}ZYZ decompose}

\input "Input/ZYZDecompose.tex"

\section{\label{sec:CSD}Cosine Sine decompose}

\input "Input/CSDecompose.tex"

\section{\label{sec:SuperpositionDecompose}Superposition matrix decompose}

\section{\label{sec:AmplitudeEncode}Amplitude Encode}

\input "Input/AmplitudeEncode.tex"

\chapter{\label{sec:Algorithm}Algorithm}

\section{\label{sec:hhl}HHL Algorithm}

\input "Algorithm/hhl.tex"

\section{\label{sec:swap}Swap test}

\input "Algorithm/swap.tex"

\section{\label{sec:grover}Grover}

\input "Algorithm/Grover.tex"

\clearpage

\printindex

\bibliography{detail}
\bibliographystyle{hunsrt}

\end{document}
%
% ****** End of file template.aps ******
