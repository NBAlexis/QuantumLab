hhl是用来解方程的。其简要步骤如下:

首先我们要求的是方程
\begin{equation}
\begin{split}
&Ax=b\\
\end{split}
\end{equation}

我们要算的就是$x=A^{-1}b$。

由于
\begin{equation}
\begin{split}
&A=\sum _i\lambda _i |\psi _i \rangle \langle \psi _i |\\
&A^{-1}=\sum _i\lambda _i^{-1} |\psi _i \rangle \langle \psi _i |\\
\end{split}
\end{equation}
我们要算的实际上是
\begin{equation}
\begin{split}
&x=\sum _i\lambda _i^{-1} |\psi _i \rangle \langle \psi _i |b\rangle\\
\end{split}
\end{equation}
这里$\langle \psi _i |b\rangle$就是$b$用$\psi _i$展开的展开系数,记为$b_i=\langle \psi _i |b\rangle$,$|b\rangle = \sum _i b_i |\psi _i\rangle$。

具体方法要用到QPE,精确的QPE的效果是,对于态$|0 \rangle$
\begin{equation}
\begin{split}
&QPE(H,t) |b\rangle \otimes |0 \rangle = QPE(H) \sum _i b_i |\psi _i\rangle \otimes |0 \rangle \\
&=\sum _i b_i |\psi _i\rangle \otimes | t\lambda _i \rangle \\
\end{split}
\end{equation}

不精确的QPE,那么假定$t\lambda _i$缩放到$0 - 2\pi$之间,假定$n$个qubit的QPE,那么$t\lambda _i$对应着距离$2^n \times t\lambda _i / 2\pi$最近的那个整数。

举个例子,比如$H$的本征值有4个,分别是$0.1, 0.3, 0.5, 0.9$,然后$t=2\pi$,$n=5$,那么$ |t\lambda _i \rangle$对应着4个态$\lambda _1 =|3\rangle= |00011\rangle$,$\lambda _2 = |10\rangle=|01010\rangle$,$\lambda _3 =|16\rangle= |10000\rangle$,$\lambda _4 = |29\rangle = |11101\rangle$。

最后,再对每个可能的QPE的量子比特对应的值做旋转,转一个辅助比特:
\begin{equation}
\begin{split}
&EIV \sum _k  | k \rangle |0\rangle = \sum _k  | k \rangle \otimes \left(\sqrt{1-\frac{C^2}{k^2}}|0\rangle + \frac{C}{k}|1\rangle\right)\\
\end{split}
\end{equation}

如果QPE成功的话,只剩下了$\lambda _i$对应的态,那么应当有
\begin{equation}
\begin{split}
&EIV \sum _i b_i |\psi _i\rangle \otimes | t\lambda _i \rangle \otimes |0\rangle = \sum _i b_i |\psi _i\rangle \otimes | t\lambda _i \rangle \otimes \left(\sqrt{1-\frac{C^2}{t^2\lambda _i^2}}|0\rangle + \frac{C}{t\lambda _i}|1\rangle\right)\\
\end{split}
\end{equation}

最后对辅助比特做测量,如果测到$1$,那么
\begin{equation}
\begin{split}
&\sum _i b_i |\psi _i\rangle \otimes | t\lambda _i \rangle \otimes \left(\sqrt{1-\frac{C^2}{t^2\lambda _i^2}}|0\rangle + \frac{C}{t\lambda _i}|1\rangle\right)\\
&\to \frac{C}{t\lambda _i}\sum _i b_i |\psi _i\rangle \otimes | t\lambda _i \rangle \otimes |1\rangle \\
\end{split}
\end{equation}

可以看到这时原来$|b\rangle$的qubit对应的正好是想要的,但是还有$|t\lambda _i\rangle$线性组合,我们可以把它们转回来
\begin{equation}
\begin{split}
&QPE^{\dagger} \frac{C}{t\lambda _i}\sum _i b_i |\psi _i\rangle \otimes | t\lambda _i \rangle \otimes |1\rangle \\
&=\frac{C}{t\lambda _i}\sum _i b_i |\psi _i\rangle \otimes | 0 \rangle \otimes |1\rangle \\
\end{split}
\end{equation}

这样,原来的$|b\rangle$的qubit上就得到了结果。

需要注意的是,为了用到$1/\lambda$的旋转,我们只能允许$0<t\lambda<2\pi$。
此外,为了允许$\lambda <0$,我们实际上要求$t\lambda<\pi$,这样,对于$t\lambda > \pi$的eigen value,直接当成负数即可。

\subsection{\label{sec:hhl.opqpe}QPE optimize}

对QPE的优化。

\subsection{\label{sec:hhl.opqpe}Rotation optimize}

对受控旋转的优化。


