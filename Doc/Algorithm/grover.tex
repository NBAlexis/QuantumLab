Grover算法和amplitude amplification。


\subsection{原理}

假设有一个circuit可以生产
\begin{equation}
\begin{split}
&A|0\rangle _a|00\rangle _b = |\phi _1\rangle _a |00\rangle _b+|\phi _2\rangle _a |10\rangle _b+|\phi _3\rangle _a |01\rangle _b+|\phi _4\rangle _a |11\rangle _b\\
\end{split}
\end{equation}
而,我们想要的是
\begin{equation}
\begin{split}
&|\phi _1\rangle _a |00\rangle _b\\
\end{split}
\end{equation}

第一步,另
\begin{equation}
\begin{split}
&|\psi _ 1\rangle  = |\phi _1\rangle _a |00\rangle _b\\
&|\psi _ 0\rangle  = |\phi _2\rangle _a |10\rangle _b+|\phi _3\rangle _a |01\rangle _b+|\phi _4\rangle _a |11\rangle _b\\
\end{split}
\end{equation}
有$\langle \psi _0 | \psi _1 \rangle = 0$,它们是正交的。


我们可以把$|\psi\rangle = A|0\rangle _a|00\rangle _b$重新写成,
\begin{equation}
\begin{split}
&A|0\rangle _a|00\rangle _b =|\psi\rangle= a|\psi _ 1\rangle+b |\psi _ 0\rangle\\
&=\sin(\theta)|\psi _ 1\rangle + \cos (\theta)|\psi _ 0\rangle\\
\end{split}
\end{equation}

另
\begin{equation}
\begin{split}
&S_{\psi}=1-2|\psi\rangle \langle \psi|\\
&S_0^b=1-2|00\rangle \langle 00|\\
\end{split}
\end{equation}

\begin{equation}
\begin{split}
&S_0^b|\psi\rangle=-\sin(\theta)|\psi _1\rangle + \cos (\theta) |\psi _0\rangle\\
&S_{\psi}S_0^b|\psi\rangle=-\sin(\theta)|\psi _1\rangle + \cos (\theta) |\psi _0\rangle -2|\psi\rangle \langle \psi|\left(-\sin(\theta)|\psi _1\rangle + \cos (\theta) |\psi _0\rangle\right)\\
&=-\sin(\theta)|\psi _1\rangle + \cos (\theta) |\psi _0\rangle -2|\psi\rangle \left(-\sin^2(\theta) + \cos^2 (\theta) \right)\\
&=-\sin(3\theta)|\psi _1\rangle - \cos(3\theta) |\psi _0\rangle\\
\end{split}
\end{equation}

更一般的情况,如果$Q=-S_{\psi}S_0^b$,那么
\begin{equation}
\begin{split}
&Q^n |\psi\rangle = \sin((2n+1)\theta)|\psi _1\rangle + \cos((2n+1)\theta) |\psi _0\rangle\\
\end{split}
\end{equation}

注意到,这个过程中,我们没有破坏$\psi _1$。


\subsection{如何实现$S_0$}

\subsection{如何实现$S_{\psi}$}

\begin{equation}
\begin{split}
&1-2|psi\rangle \langle \psi | = AA^{\dagger} - 2 A|0\rangle \langle 0|A^{\dagger}\\
&=A(1-2|0\rangle \langle 0 |)A^{\dagger}\\
\end{split}
\end{equation}

\subsection{Oracle}

Oracle的作用是实现
\begin{equation}
\begin{split}
&f(x)=\left\{\begin{array}{cc}-1 & x=n\\ 1 & x\neq n \end{array}\right.
\end{split}
\end{equation}

这个可以通过一个phase kickback在一个辅助bit上实现。假设$f(n)$对应一个Cn-NOT门,且,
\begin{equation}
\begin{split}
&f(n)|x\rangle |\phi\rangle =\left\{\begin{array}{cc}  \sigma _x |x\rangle |\phi\rangle & x=n\\ |x\rangle|\phi\rangle & x\neq n \end{array}\right.
\end{split}
\end{equation}
那么
\begin{equation}
\begin{split}
&f(n)|x\rangle \frac{|0\rangle -|1\rangle}{\sqrt{2}} =\left\{\begin{array}{cc}  -|x\rangle \frac{|0\rangle -|1\rangle}{\sqrt{2}} & x=n\\ |x\rangle \frac{|0\rangle -|1\rangle}{\sqrt{2}} & x\neq n \end{array}\right.
\end{split}
\end{equation}

注意$f(x)=1-2|x\rangle \langle x|$
