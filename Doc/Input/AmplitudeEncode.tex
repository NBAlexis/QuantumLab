Amplitude Encode,针对的是,当有一个矢量
\begin{equation}
\begin{split}
&v=\left(v_1,v_2,v_3,\ldots\right),\;\; v^{\dagger}v=1\\
\end{split}
\end{equation}
如何得到
\begin{equation}
\begin{split}
&| \Psi \rangle = \sum v_i |i\rangle \\
\end{split}
\end{equation}

\subsection{\label{sec:AESVD}原理}

第一步,我们需要使用SVD。或者称为Schmidt decomposition。

为了方便举例,我们使用20个数字。由于$2^4 < 20 < 2^5$,可以预见需要5个qubit。


\begin{equation}
\begin{split}
&v=\{0.0889849\, +0.00187365 i,-0.0224399-0.177007 i,-0.180764-0.193077 i,-0.0958356-0.176877 i,\\
&0.0780797\, -0.170434 i, 0.210302\, -0.0292187 i,-0.0824832-0.148132 i,-0.107826-0.187449 i, \\
&0.200649\, -0.112298 i, -0.207594-0.213168 i,-0.122852+0.254445 i,-0.241442+0.171078 i,\\
&0.248816\, +0.198027 i,-0.0571302-0.0122686 i,0.239673\, -0.072375 i,-0.138341-0.103959 i,\\
&0.152874\, +0.0900167 i,0.109308\, -0.186814 i,-0.0700675-0.134879 i,-0.222134+0.193257 i\}\\
\end{split}
\end{equation}

第一步,我们需要将矢量补充到5个qubit,也就是32个数字,然后写成$2^2 \times 2^3$矩阵:
\begin{equation}
\begin{split}
    M=
\left(
\begin{array}{cccc}
 0.0889849\, +0.00187365 i & -0.0224399-0.177007 i & -0.180764-0.193077 i & -0.0958356-0.176877 i \\
 0.0780797\, -0.170434 i & 0.210302\, -0.0292187 i & -0.0824832-0.148132 i & -0.107826-0.187449 i \\
 0.200649\, -0.112298 i & -0.207594-0.213168 i & -0.122852+0.254445 i & -0.241442+0.171078 i \\
 0.248816\, +0.198027 i & -0.0571302-0.0122686 i & 0.239673\, -0.072375 i & -0.138341-0.103959 i \\
 0.152874\, +0.0900167 i & 0.109308\, -0.186814 i & -0.0700675-0.134879 i & -0.222134+0.193257 i \\
 0 & 0 & 0 & 0 \\
 0 & 0 & 0 & 0 \\
 0 & 0 & 0 & 0 \\
\end{array}
\right)
\end{split}
\end{equation}

第三步,使用SVD(这里直接用Mathematica的SingularValueDecomposition),
\begin{equation}
\begin{split}
&M=USV^{\dagger}
\end{split}
\end{equation}
其中
\begin{equation}
\begin{split}
&U=\left(
    \begin{array}{cccccccc}
     -0.396859-0.142579 i & -0.0891902-0.128741 i & 0.445746\, +0.0360418 i & 0.631789\, -0.0144329 i & 0.324444\, +0.305004 i & 0.\, +0. i & 0.\, +0. i & 0.\, +0. i \\
     -0.279024-0.24102 i & -0.0709793-0.41251 i & -0.0202485-0.412174 i & -0.307046+0.476123 i & 0.335931\, -0.291094 i & 0.\, +0. i & 0.\, +0. i & 0.\, +0. i \\
     -0.356398+0.553047 i & 0.161774\, +0.30132 i & 0.157339\, -0.228711 i & -0.252355+0.341162 i & -0.112324+0.424735 i & 0.\, +0. i & 0.\, +0. i & 0.\, +0. i \\
     -0.106766-0.229524 i & 0.633013\, +0.462781 i & 0.129936\, -0.140917 i & 0.23816\, +0.120619 i & -0.0655714-0.456879 i & 0.\, +0. i & 0.\, +0. i & 0.\, +0. i \\
     -0.434158+0.0278483 i & 0.253322\, +0.0652703 i & -0.523895+0.491368 i & -0.127401-0.109956 i & 0.442766\, +0.0450247 i & 0.\, +0. i & 0.\, +0. i & 0.\, +0. i \\
     0.\, +0. i & 0.\, +0. i & 0.\, +0. i & 0.\, +0. i & 0.\, +0. i & 1.\, +0. i & 0.\, +0. i & 0.\, +0. i \\
     0.\, +0. i & 0.\, +0. i & 0.\, +0. i & 0.\, +0. i & 0.\, +0. i & 0.\, +0. i & 1.\, +0. i & 0.\, +0. i \\
     0.\, +0. i & 0.\, +0. i & 0.\, +0. i & 0.\, +0. i & 0.\, +0. i & 0.\, +0. i & 0.\, +0. i & 1.\, +0. i \\
    \end{array}
    \right)
\end{split}
\end{equation}

\begin{equation}
\begin{split}
&S=\left(
    \begin{array}{cccc}
     0.783244 & 0. & 0. & 0. \\
     0. & 0.496363 & 0. & 0. \\
     0. & 0. & 0.323968 & 0. \\
     0. & 0. & 0. & 0.187609 \\
     0. & 0. & 0. & 0. \\
     0. & 0. & 0. & 0. \\
     0. & 0. & 0. & 0. \\
     0. & 0. & 0. & 0. \\
    \end{array}
    \right)
\end{split}
\end{equation}

\begin{equation}
\begin{split}
&V=\left(
    \begin{array}{cccc}
     -0.364878 & 0.703027 & 0.414297 & -0.448302 \\
     -0.134242-0.488901 i & -0.205988-0.189388 i & -0.454553+0.1152 i & -0.633844+0.207385 i \\
     0.45985\, -0.220635 i & 0.516566\, +0.277431 i & -0.279526-0.0952313 i & 0.177479\, +0.526637 i \\
     0.586835\, -0.0728136 i & -0.101846-0.270689 i & 0.49772\, +0.52301 i & -0.177379+0.118108 i \\
    \end{array}
    \right)
\end{split}
\end{equation}

注意这里$U$的右边4列其实根本用不到。


最后,使用(注意是$U^T$不是$U^{\dagger}$)
\begin{equation}
\begin{split}
&\begin{pmatrix} \psi _1 \\ \psi _2 \\ \psi _3 \\ \psi _4 \\ nouse \\ nouse \\ nouse \\ nouse \end{pmatrix}= U^T\cdot \begin{pmatrix} a000 \\ a001 \\ a010 \\ a011 \\ a100 \\ a101 \\ a110 \\ a111 \end{pmatrix}\\
&\begin{pmatrix} \phi _1 \\ \phi _2 \\ \phi _3 \\ \phi _4 \end{pmatrix}= V^{\dagger}\cdot \begin{pmatrix} b00 \\ b01 \\ b10 \\ b11  \end{pmatrix}\\
&|\Psi\rangle = \sum _{i=1}^4 s_{ii} \psi _i \phi _i\\
\end{split}
\end{equation}

经过计算
\begin{equation}
\begin{split}
&\Psi = (0.0889849\, +0.00187365 i) \text{a000} \text{b00}-(0.0224399\, +0.177007 i) \text{a000} \text{b01}\\
&-(0.180764\, +0.193077 i) \text{a000} \text{b10}-(0.0958356\, +0.176877 i) \text{a000} \text{b11}\\
&+(0.0780797\, -0.170434 i) \text{a001} \text{b00}+(0.210302\, -0.0292187 i) \text{a001} \text{b01}\\
&-(0.0824832\, +0.148132 i) \text{a001} \text{b10}-(0.107826\, +0.187449 i) \text{a001} \text{b11}\\
&+(0.200649\, -0.112298 i) \text{a010} \text{b00}-(0.207594\, +0.213168 i) \text{a010} \text{b01}\\
&-(0.122852\, -0.254445 i) \text{a010} \text{b10}-(0.241442\, -0.171078 i) \text{a010} \text{b11}\\
&+(0.248816\, +0.198027 i) \text{a011} \text{b00}-(0.0571302\, +0.0122686 i) \text{a011} \text{b01}\\
&+(0.239673\, -0.072375 i) \text{a011} \text{b10}-(0.138341\, +0.103959 i) \text{a011} \text{b11}\\
&+(0.152874\, +0.0900167 i) \text{a100} \text{b00}+(0.109308\, -0.186814 i) \text{a100} \text{b01}\\
&-(0.0700675\, +0.134879 i) \text{a100} \text{b10}-(0.222134\, -0.193257 i) \text{a100} \text{b11}
\end{split}
\end{equation}

\subsection{\label{sec:AEImplement}实现}

