因为我们面对的刚好是$2^n \times 2^n$矩阵,cosine-sine decompose 十分适合。

由文献,可以看到~\cite{compareIsometries},对于方阵,CSD也是较优的分解方式。

\subsection{\label{sec:ucrgate}Uniform controlled rotation gate}

所谓``Uniform controlled rotation gate'',就是
\begin{equation}
\begin{split}
&F_m^k({\bf a},\{\theta _i\}) \sum _{i}^k|i\rangle \otimes \begin{pmatrix} \uparrow \\ \downarrow \end{pmatrix}_m =\sum _{i}^k |i\rangle \otimes R_{\bf a}(\theta _i)\begin{pmatrix} \uparrow \\ \downarrow \end{pmatrix}_m
\end{split}
\end{equation}
也就是,对于$m$的qubit,根据它前面的控制$i$的qubit施加不同的转动。${\bf a}=y,z$,通常不做$R_x$旋转。

\subsection{\label{sec:CSD}Cosine-sine decompose 的原理}

CSD的原理是
\begin{equation}
\begin{split}
&U=\begin{pmatrix}U^1_{11} & 0 \\ 0 & U^1_{12} \\ \end{pmatrix}\cdot\begin{pmatrix}C^1_{11} & S^1_{11} \\ -S^1_{11} & C^1_{11} \\ \end{pmatrix}\cdot\begin{pmatrix}U^1_{21} & 0 \\ 0 & U^1_{22} \\ \end{pmatrix}
\end{split}
\end{equation}
其中
$C$,$S$是$2^{n-1}\times 2^{n-1}$的diagnal matrix。
这其中,$C$,$S$还可以写为
\begin{equation}
\begin{split}
&C={\rm diag}\left\{\cos (\theta _1), \cos (\theta _2),\ldots \right\}\\
&S={\rm diag}\left\{\sin (\theta _1), \sin (\theta _2),\ldots \right\}\\
\end{split}
\end{equation}
$U^i_{mn}$表示第$i$次分解的,位于第$m$个$U$的第$n$个对角线位置的幺正矩阵。

因为
\begin{equation}
\begin{split}
&\begin{pmatrix}A_1 & 0 \\ 0 & B_1 \\ \end{pmatrix}\cdot \begin{pmatrix}A_2 & 0 \\ 0 & B_2 \\ \end{pmatrix}\cdot \begin{pmatrix}A_3 & 0 \\ 0 & B_3 \\ \end{pmatrix}= \begin{pmatrix}A_1\cdot A_2\cdot A_3 & 0 \\ 0 & B_1\cdot B_2 \cdot B_3\\ \end{pmatrix}
\end{split}
\end{equation}

上面的分解可以进一步写为
\begin{equation}
\begin{split}
&U=\begin{pmatrix}U^2_{11} & 0 & 0 & 0\\ 0 & U^2_{12} & 0 & 0 \\ 0 & 0 & U^2_{13} & 0 \\ 0 & 0 & 0 & U^2_{14} \end{pmatrix} 
  \cdot\begin{pmatrix}C^2_{11} & S^2_{11} & 0 & 0\\ -S^2_{11} & C^2_{11} & 0 & 0 \\ 0 & 0 & C^2_{12} & S^2_{12} \\ 0 & 0 & -S^2_{12} & C^2_{12} \end{pmatrix}
  \cdot \begin{pmatrix}U^2_{21} & 0 & 0 & 0\\ 0 & U^2_{22} & 0 & 0 \\ 0 & 0 & U^2_{23} & 0 \\ 0 & 0 & 0 & U^2_{24} \end{pmatrix} \\
& \cdot \begin{pmatrix}C^1_{11} & S^1_{11} \\ -S^1_{11} & C^1_{11} \\ \end{pmatrix} \\
& \begin{pmatrix}U^2_{31} & 0 & 0 & 0\\ 0 & U^2_{32} & 0 & 0 \\ 0 & 0 & U^2_{33} & 0 \\ 0 & 0 & 0 & U^2_{34} \end{pmatrix} 
\cdot\begin{pmatrix}C^2_{21} & S^2_{21} & 0 & 0\\ -S^2_{21} & C^2_{21} & 0 & 0 \\ 0 & 0 & C^2_{22} & S^2_{22} \\ 0 & 0 & -S^2_{22} & C^2_{22} \end{pmatrix}
\cdot \begin{pmatrix}U^2_{41} & 0 & 0 & 0\\ 0 & U^2_{42} & 0 & 0 \\ 0 & 0 & U^2_{43} & 0 \\ 0 & 0 & 0 & U^2_{44} \end{pmatrix} \\
\end{split}
\end{equation}

进一步的:
\begin{equation}
\begin{split}
&U=U^3_1 A^3_1 U^3_2 \cdot A^2_1 \cdot U^3_3 A^3_2 U^3_4 \cdot A^1_1 \cdot U^3_5  A^3_3 U^3_6 \cdot A^2_2 \cdot U^3_7 A^3_4 U^3_8\\
\end{split}
\end{equation}

接下来,我们在每个$A^i_j$右侧插入能与$A$对易,且对角的,幺正的矩阵,$P^i_j$,同时,要求$A$左侧的$U$被转换成``uniform controlled rotation z''。

这件事总是能做到的。首先,对于$A^i_j$,它是一个``uniform controlled rotation y''。

然后,我们可以选择$P$为,对旋转的那个qubit加一个整体相位。这样就是与$A$对易的。具体来说:
\begin{equation}
\begin{split}
&A=\begin{pmatrix} c_1 & 0 & s_1 & 0 \\ 0 & c_2 & 0 & s_2 \\ -s_1 & 0 & c_1 & 0 \\ 0 & -s_2 & 0 & c_2\end{pmatrix} 
\to P= \begin{pmatrix} d_1 & 0 & 0 & 0 \\ 0 & d_2 & 0 & 0 \\ 0 & 0 & d_1 & 0 \\ 0 & 0 & 0 & d_2 \end{pmatrix}\\
&A=\begin{pmatrix} c_1 & s_1 & 0 & 0 \\ -s_1 & c_1 & 0 & 0 \\ 0 & 0 & c_2 & s_2 \\ 0 & 0 & -s_2 & c_2\end{pmatrix} 
\to P= \begin{pmatrix} d_1 & 0 & 0 & 0 \\ 0 & d_1 & 0 & 0 \\ 0 & 0 & d_2 & 0 \\ 0 & 0 & 0 & d_2 \end{pmatrix}\\
\end{split}
\end{equation}

我们可以选择合适的$d_1,d_2$,把$A^i_j$左边的矩阵变成和$A$相同对象的``uniform controlled rotation z''。

这一步可以一直递归直到最后一个矩阵。也就是
\begin{equation}
\begin{split}
&U=U^3_1 A^3_1 U^3_2 \cdot A^2_1 \cdot U^3_3 A^3_2 U^3_4 \cdot A^1_1 \cdot U^3_5  A^3_3 U^3_6 \cdot A^2_2 \cdot U^3_7 A^3_4 U^3_8\\
&=U^3_1 F_3(R_y) U^3_2 \cdot F_2(R_y) \cdot U^3_3 F_3(R_y) U^3_4 \cdot F_1(R_y) \cdot U^3_5  F_3(R_y) U^3_6 \cdot F_2(R_y) \cdot U^3_7 F_3(R_y) U^3_8\\
&=U^3_1 F_3(R_y) P_1 P_1^{\dagger}U^3_2 \cdot F_2(R_y) \cdot U^3_3 F_3(R_y) U^3_4 \cdot F_1(R_y) \cdot U^3_5  F_3(R_y) U^3_6 \cdot F_2(R_y) \cdot U^3_7 F_3(R_y) U^3_8\\
&=U^3_1 P_1 F_3(R_y) (P_1^{\dagger}U^3_2) \cdot F_2(R_y) \cdot U^3_3 F_3(R_y) U^3_4 \cdot F_1(R_y) \cdot U^3_5  F_3(R_y) U^3_6 \cdot F_2(R_y) \cdot U^3_7 F_3(R_y) U^3_8\\
&=F_3(R_z) F_3(R_y) (P_1^{\dagger}U^3_2) \cdot F_2(R_y) \cdot U^3_3 F_3(R_y) U^3_4 \cdot F_1(R_y) \cdot U^3_5  F_3(R_y) U^3_6 \cdot F_2(R_y) \cdot U^3_7 F_3(R_y) U^3_8\\
&=F_3(R_z) F_3(R_y) (P_1^{\dagger}U^3_2) \cdot F_2(R_y) P_2 P_2^{\dagger}\cdot U^3_3 F_3(R_y) U^3_4 \cdot F_1(R_y) \cdot U^3_5  F_3(R_y) U^3_6 \cdot F_2(R_y) \cdot U^3_7 F_3(R_y) U^3_8\\
&=F_3(R_z) F_3(R_y) F_2(R_z) \cdot F_2(R_y) (P_2^{\dagger}\cdot U^3_3) F_3(R_y) U^3_4 \cdot F_1(R_y) \cdot U^3_5  F_3(R_y) U^3_6 \cdot F_2(R_y) \cdot U^3_7 F_3(R_y) U^3_8\\
&=F_3(R_z) F_3(R_y) F_2(R_z) \cdot F_2(R_y) F_3(R_z) F_3(R_y) (P_3^{\dagger}U^3_4) \cdot F_1(R_y) \cdot U^3_5  F_3(R_y) U^3_6 \cdot F_2(R_y) \cdot U^3_7 F_3(R_y) U^3_8\\
&\ldots\\
&=\prod \left(F(R_z) F(R_y)\right) P_8^{\dagger}U^3_8\\
\end{split}
\end{equation}
最后的$P_8^{\dagger}U^3_8$只能用$1$个$R_z$加$2$个uniform controlled $R_z$实现。具体实现方案考虑3个qubit的情况,对全部的qubit转$a_0$,对一个qubit转$R_z(a_1)$,对二个qubit转$FR_z(a_{21}, a_{22})$,对三个qubit转$FR_z(a_{31}, a_{32}, a_{33}, a_{34})$
\begin{equation}
\begin{split}
&a_0+a_1+a_{21}+a_{31}=b_1\\
&a_0+a_1+a_{21}-a_{31}=b_2\\
&a_0+a_1-a_{21}+a_{32}=b_3\\
&a_0+a_1-a_{21}-a_{32}=b_4\\
&a_0-a_1+a_{22}+a_{33}=b_5\\
&a_0-a_1+a_{22}-a_{33}=b_6\\
&a_0-a_1-a_{22}+a_{34}=b_7\\
&a_0-a_1-a_{22}-a_{34}=b_8\\
\end{split}
\end{equation}

这就是文献~\cite{CSDecompose}的方案。




